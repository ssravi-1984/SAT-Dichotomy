\section{Definitions and Preliminaries}\label{sec:def}

Subsections  \ref{sec:other_def} through \ref{sec:fpn_spec} give
basic definitions used here, including the definitions of the
generalized {\sf CNF} Satisfiability problems  of \cite{Sc78} and
the kinds of succinct specifications considered. We illustrate these
definitions with several examples.


%%\vspace*{-0.15in}
\subsection{The Problems {\sf SAT(S)}}\label{sec:other_def}

We first review some basic definitions  from Schaefer \cite{Sc78}.

\begin{definition}\label{s-formulas:def}(Schaefer \cite{Sc78})\\
Let S $= \{R_1,R_2, \cdots, R_m \}$ be a finite 
set of finite arity Boolean
relations. (A Boolean relation is defined to be 
any subset of $\{0,1 \}^p$ for some integer $p \geq 1$. 
The integer $p$ is called the  {\bf rank} of the relation.)
An S-{\it formula} is a conjunction of clauses each of the form 
$\hat{R}_i(\xi_1, \xi_2, \cdots)$, where $\xi_1, \xi_2, \cdots$ are distinct, 
unnegated variables whose number matches the 
rank of $R_i, i \in \{ 1, \cdots m \}$ and $\hat{R}_i$ is the relation symbol
representing the relation $R_i$.
The  S-{\it Satisfiability problem} (denoted by SAT(S)) 
is the problem of deciding 
whether a given S-formula is satisfiable.

The problem {\sf SAT$_c$(S)} is the variant of the problem 
{\sf SAT(S)} in which the constants $0$ and $1$ are also allowed
to occur in {\sf S-}formulas. 
\end{definition}


The 
generalized {\sf CNF} Satisfiability problems {\sf SAT(S)} and 
{\sf SAT$_c$(S)} generalize  the problem  {\sf 3SAT} and its variants
studied in the literature \cite{GJ79}. For example,
let $R(x,y,z)$ be the ternary logical relation  given by
$\{(1,0,0), (0,1,0), (0,0,1) \}$. Then, the problem 
{\sf 1-3SAT}  is the same as the problem {\sf SAT(\{R\})}. 

\iffalse
Similarly, {\sf NAE3SAT} is the problem of determining if
a {\sf 3CNF} formula, has a satisfying assignment in which all the literals
in any clause are not simultaneously true. 
\fi



\begin{definition}\label{def:affine}

\begin{enumerate}
\item
The logical relation $R$ is {\bf 0-valid} if $(0, \ldots, 0) \in R$.
\item
The logical relation $R$ is {\bf 1-valid} if $(1, \ldots, 1) \in R$.
\item
The logical relation $R$ is {\bf weakly positive} if
$R(x_1,x_2,\ldots)$ is logically equivalent to some CNF formula having at
most one negated variable in each conjunct.
\item
The logical relation $R$ is {\bf weakly negative} if
$R(x_1,x_2,\ldots)$ is logically equivalent to some CNF formula having at
most one unnegated variable in each conjunct.
\item
The logical relation $R$ is {\bf bijunctive} if
$R(x_1,x_2,\ldots)$ is logically equivalent to some CNF formula having at
most two literals in each conjunct.
\item
The logical relation $R$ is {\bf affine} if $R(x_1,x_2,\ldots)$ is logically 
equivalent to some system of linear equations over the two-element field 
${\bf Z}_2$.
\item
The logical relation $R$ is {\bf complementive} 
if it is closed under component-wise
complement; i.e., for all $(a_1,\ldots,a_m)$, if $(a_1,\ldots,a_m) \in R$, 
then $(1-a_1,\ldots,1-a_m) \in R$.
\item
The logical relation $R$ is {\bf 1-weakly positive} if
$R(x_1,x_2,\ldots)$ is logically equivalent to some CNF formula having at
most one negated variable in each conjunct, such that any clause with
a negated literal has no more than 1 unnegated literal.
\item
The logical relation $R$ is {\bf 1-weakly negative} if
$R(x_1,x_2,\ldots)$ is logically equivalent to some CNF formula having at
most one unnegated variable in each conjunct, such that any clause with
an unnegated literal has no more than 1 negated literal.
\end{enumerate}
\end{definition}



\begin{definition}\label{def:rep}
(1)~ If $A$ is a formula, then we denote the 
logical relation defined by $A$ as {\bf $[A]$},
when the variables are taken in ascending order.

\noindent
(2)~ {\sf $Gen(S)$} is the set of existentially quantified $S$-formulas with
constants.

\noindent
(3)~ {\sf $Rep(S)$} is the set of relations that are representable by existentially 
quantified $S$-formulas with constants.

\noindent
(4)~ {\sf $Rep_{NC}(S)$} is the set of relations that are 
representable by existentially quantified $S$-formulas without constants.
\end{definition}




%%\vspace*{-0.15in}
\subsection{The L-Specified Formulas}\label{sec:lspec}
We define the terminology for {\sf L-}specified generalized {\sf CNF} formulas used here. 
%\subsubsection{L-specified S formulas}\label{sec:hsat_def}
Let $S$ be a finite nonempty set of finite-arity Boolean 
relations. The definitions of L-specification of an S-formula, the 
S-formulas specified by such specifications, and the problems 
{\sf L-SAT(S)} and {\sf L-SAT}$_c${\sf (S)} are closely analogous to Definitions
3.5, 3.6 and 3.1, respectively.

\begin{definition}
Let $X$ be a vector of Boolean variables and the constants $0$ and $1$. Then
$\hat{X}$ is the set of all variables occurring in $X.$
\end{definition}

\begin{definition}\label{hsats:def}
An {\sf L-}specification $F(X) = (A_1(X_1),\ldots, A_n(X_n))\, (n \geq 1)$ 
of an S-formula (or of an S-formula with constants) is a finite nonempty 
sequence of assignment statements $A_i(X_i)$ of the form 
\[ F_i(X_i) \leftarrow 
(\bigwedge_{1 \leq k \leq  l_i} F_{i_k}(X_{i_k},Z_{i_k})) \bigwedge 
f_i(X_i,Z_i), \]
where, for $1 \leq i_k \leq l_i$,\\
(1) $1 \leq i_k < i,$ \\
(2) $X_i, X_{i_k}, Z_i$ and $Z_{i_k}$ are vectors of Boolean variables (or are 
vectors of Boolean variables, $0$, and $1$) respectively; and \\
(3) $X = \hat{X},~ \hat{X}_{i_k} \subseteq \hat{X_i},$ ~and~ $ \hat{Z_{i_k}} 
\subseteq \hat{Z_i}$. 

\smallskip

\noindent
For $1 \leq i \leq n$, letting $\lambda_i$ be the number of occurrences of
literals in the statement $A_i$, the {\em size} of $F(X)$ denoted 
{\em size($F(X))$}, equals $\sum_{i=1}^{n} \lambda_i + n$.
\end{definition}

Given Definition \ref{hsats:def}, the definition of the 
{\sf S-formula} (or of the {\sf S}-formula with constants) 
$E(F(X))$ specified
by such an  L-specification $F(X)$ very closely follows Definition 3.6 and 
is not repeated here. 





\begin{example}
Let $F= (F_1(x_1,x_2),F_{2}(x_3,x_4), F_3)$
be an instance of {\sf L-3SAT} where each $F_i$ is defined as follows:
\[F_1(x_1,x_2) = (x_1 + x_2 + z_1) \wedge (z_2 + z_3) \]
\[F_2(x_3,x_4) = F_1(x_3, z_4) \wedge F_1(z_4, z_5) \wedge
                               (z_4 + z_5 + x_4)   \]
\[F_3 = F_2(z_8,z_7) \wedge  F_1(z_7,z_6) \]
The formula $E(F)$ denoted by $F$ is given by 
$ (z_7 + z_6 + z_1^1) \wedge (z_2^1 + z_3^1) \wedge
(z_8 + z_4 + z_1^2) \wedge (z_2^2 + z_3^2) \wedge $
$(z_4 + z_5 + z_1^3) \wedge (z_2^3 + z_3^3) \wedge (z_4 + z_5 + z_7)$. 

\end{example}


\begin{definition}
The problem {\sf L-SAT(S)} (or {\sf L-SAT}$_c${\sf (S)}) is the
problem of determining, given an L-specification $F(X)$ of the S-formula(or of
the S-formula with constants), if ${\cal F}(F(X))$ is satisfiable.
\end{definition}
 






%%\vspace*{-0.15in}
\subsection{Periodic Specifications }\label{sec:fpn_spec}
Next, we recall the definition of one dimensional periodic specifications 
due to  Orlin \cite{Or82a}, Wanke \cite{Wa93} and 
H\"ofting and Wanke \cite{HW92}. 
In the remainder of this paper, {\bf N} and {\bf Z} re used to 
denote the sets of non-negative integers and integers respectively.



%%\vspace*{-0.15in}
\subsubsection{Periodically Specified Formulas}

Earlier in this section, we extended
the definition of Satisfiability problems when instances are specified
by standard specifications so as to apply to 
{\sf L}-specified Satisfiability problems. Orlin \cite{Or82a} 
defined the problem {\sf 1-PN-3SAT}, that is, the problem {\sf 3SAT} when 
instances are specified using {\sf 1-PN}-specifications.
Here we define {\sf 1-FPN}- and {\sf 1-PN}-specified generalized Satisfiability
problems.




Let $U = \{u_1, \ldots, u_n \}$  be a finite set of variables (referred to
as static variables).
Further, let
$U^{\infty} =  \{u_k(i): ~ 1 \leq k \leq n, ~ i \in {\bf Z}\}$ and 
$U^{M} =  \{u_k(i): ~ 1 \leq k \leq n, ~ i \in \{0,1, 2, \cdots, M\}\}$. 
In our proofs, variable $u_k(i)$
denotes the variable $u_k$ at time unit $i$. 
A literal of $U$ is an element of 
$\{u_1, \ldots, u_n, \overline{u_1}, \ldots, \overline{u_n} \}$.
If $w$ is a literal of $U$, then $w(i)$, 
is a literal of $U^{\infty}$.
Let $C(i, i+1)$ be a parameterized conjunction of 3 literal clauses
such that each clause in $C(i, i+1)$ 
consists of variables $u_k(i), u_k(i+1)$ with the constraint that at least one
variable is of the form $u_k(i)$. 
We refer to the clauses $C(i, i+1)$ as  {\em static narrow clauses}. 
($C(i, i+1)$ is called  narrow because
for all  $(w_1(i_1) \vee w_2(i_2) \vee w_3(i_3)) \in C(i,i+1) $, 
$|i_s - i_r| \leq 1$ for $1 \leq r \leq s \leq 3$.)
Let $C^{\infty}$ $= \bigwedge_{i = 0}^{i =\infty}C(i, i+1)$. 
Given $U^{M}$ and $C^{\infty}$, let $C^{M}$ be a subset of $\cal C$ 
with the following property: for each clause 
$(w_1(i_1) \vee w_2(i_2) \vee w_3(i_3))  \in C^{M}$,
$ w_1(i_1), w_2(i_2), w_3(i_3) \in U^{M}$.



It is useful to imagine a narrow periodically specified formula
$\Gamma^{\infty}$ as being obtained by placing a copy of the variable 
set $U$
at each integral point (also referred to as time unit) 
on the X-axis (or the time line). 
Furthermore, assume that the clauses $C(t, t+1)$ are placed at time $t$.
With this notation, we can now refer to variables $U(t)$
as the set of variables at time $t$ 
and the clauses $C(t, t+1)$ as the set of clauses at time $t$.




\begin{definition}\label{fpn:sat}
A 1-dimensional infinite (finite) 
periodic narrow specification (denoted by 1-PN (FPN)-specification)
of a  3CNF formula  $F^{\infty}(U^{\infty}, C^{\infty})$
($F^{M}(U^{M}, C^{M})$) is given by 
($~\Gamma = (U,C(i,i+1))$,  ($~\Gamma = (U,C(i, i+1), M)$),  
where $U$ is  a finite set of variables and
$C(i, i+1)$ is a collection of static narrow 3 literal clauses. 
(In case of finite specifications $M$ is a non-negative integer 
specified in binary.)
The size of the specification denoted by 
$size(\Gamma) = |U| + |C(i, i+1)| $. (In case of finite specifications
$size(\Gamma)$ = $|U|$ $+ |C(i, i+1)|$ $+ bits(M)$,  
where $bits(M)$ denote the number
of bits used to represent $M$.)

\smallskip

The problem 1-PN-3SAT (i.e., problem 3SAT when instances are specified using 
1-FPN-specifications)  is the problem of determining if a 3CNF formula
$F^{\infty}(U^{\infty}, C^{\infty})$ specified by $\Gamma= (U,C(i, i+1))$ 
is satisfiable.

\smallskip

Similarly, the problem 1-FPN-3SAT (i.e., problem 3SAT specified using 
1-FPN-specifications)  is the problem of determining if a 3CNF formula
$F^{M}(U^{M}, C^{M})$ specified by $\Gamma$ = $(U,C(i, i+1),M)$ 
is satisfiable.
\end{definition}


For each finite set {\sf S} of finite arity Boolean relations, 
it is straightforward to extend the above 
definition so as to define the problems {\sf 1-PN-SAT(S)}  
and {\sf 1-FPN-SAT(S)};  hence, we omit those
definitions.  Observe that 
{\sf 1-FPN}-specified graphs or formulas can be exponentially larger 
than their input specifications. Finally, as in the previous subsection, 
we use ${\cal N}$ to denote the size of the formula $F^m(U^m, C^m)$ represented
using standard (non-succinct) specifications.

\smallskip
	 
\noindent
{\bf Example 4:}
Let $F= (U,C(i, i+1),3)$ be an instance of  {\sf 1-FPN-3SAT} 
where the set of static clauses are given by
$(x_1(i) + x_2(i) + x_3(i) )  \wedge 
(x_1(i+1) + x_3(i)) \wedge (x_3(i+1) + x_2(i))$.
The set of variables $U = \{ x_1, x_2, x_3 \}$.
The formula $F^3(U^3,C^3)$ denoted by $\Gamma$ is given by 
\[(x_1(0) + x_2(0) + x_3(0)) \wedge (x_1(1) + x_3(0)) \wedge (x_3(1) + x_2(0)) 
\bigwedge \]
\[(x_1(1) + x_2(1) + x_3(1)) \wedge (x_1(2) + x_3(1)) \wedge (x_3(2) + x_2(1)) 
\bigwedge \]
\[(x_1(2) + x_2(2) + x_3(2)) \wedge (x_1(3) + x_3(2)) \wedge (x_3(3) + x_2(2)) 
\bigwedge \]
\[(x_1(3) + x_2(3) + x_3(3)).\] \hfill\QED



%%\vspace*{-0.15in}
\subsubsection{1-FPN(BC)-specified Graphs and Formulas.}~
Some instances of problems arising in practice involve a periodic specification
of the graph or a formula along with explicit initial and final conditions.
We refer to such periodic specifications as periodic specifications with boundary
conditions ({\sf BC}). 
Observe that for {\sf 1-PN}-specifications boundary conditions
do not make sense since the expanded graph or the formula is infinite.
Hence boundary conditions are used to augment only {\sf 1-FPN}-specifications.

In such a case, we have a parameterized set of variables $V(i)$.
We place a copy of the variables from time $\{ 0, \cdots, m \}$.
We also have explicit variables $INIT$ and $FIN$ 
representing the variables at the
two boundaries. There are three sets of clauses.
The set of explicit clauses $C_1(INIT, V(0))$ contain variables from 
$INIT \cup V(0)$  with the constraint that at least one variable in each clause
is from the set $INIT$. Then we have the parameterized set of clauses
$C_2(i, i+1)$ which are identical to the clauses used to specify 
{\sf 1-FPN}-formulas. They contain variables from $V(i) \cup V(i+1)$.
Finally we have explicit clauses $C_3(V(m), FIN)$ which contain variables from
$FIN \cup V(m)$. 
The specification of the {\sf 1-FPN(BC)-CNF}-formula is given by
$\Gamma = ( F, m)$, where 
\[F = C_1(INIT, V(0) \cup C_2(V(i) \cup V(i+1)) \cup C_3(V(m) \cup FIN). \]
The formula specified by $\Gamma$ is given by 
\[E(\Gamma) =  
C_1(INIT, V(0)) \bigwedge \bigwedge_{i = 0}^{ i = m} C(i, i+1) \bigwedge 
C_3(V(m), FIN).  \]
