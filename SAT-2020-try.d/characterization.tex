\section{Completing the Characterization}\label{sec:char}

We are nwo ready to complete the characterization of the problems 
{\sf L-SAT(S)},
{\sf 1-FPN-SAT(S), 1-FPN(BC)-SAT(S)} and {\sf PN-SAT(S)}. 
To do this we need the  following results from Schaefer \cite{Sc78}.


\begin{theorem}\label{th:sch3.0}
Let S 
be any set of logical relations. If S satisfies one of the conditions
(a)-(d) below, then Rep(S) satisfies the same condition. Otherwise,
Rep(S) is the set of all logical relations.\\
(a) Every relation in  S is weakly positive.\\
(b) Every relation in S is weakly negative.\\
(c) Every relation in S is affine.\\
(d) Every relation in S  is bijunctive.
\end{theorem}


\begin{theorem}\label{th:sch5.1}
If every relation in S is weakly positive, and $S$ contains some relation
that is not 1-weakly positive, then Rep(S) is the set of all weakly positive
relations.
If every relation in S is weakly negative, and $S$ contains some relation
that is not 1-weakly negative, then Rep(S) is the set of all weakly negative
relations.
\end{theorem}



\begin{lemma}\label{le:sch4.3}
Let $S$ be a set of nonempty logical relations. Then at least one of the 
following holds:
\oldspacing
\begin{enumerate}
\item
Every relation in S is 0-valid.
\item
Every relation in S is 1-valid.
\item
$[x]$ and $[\overline{x}]$ are contained in Rep$_{NC}$(S).

\item
$[x \not\equiv y] \in$ Rep$_{NC}$(S).
\end{enumerate}
\end{lemma}




\begin{theorem}\label{th:hsatshard}
\begin{enumerate}
\item
If Rep(S) is the set of all finite arity boolean relations, then\\
(a) the problem 1-FPN-SAT$_c$(S) is PSPACE-complete.
(b) The problems 1-FPN-SAT(S) and 1-PN-SAT(S) are  in P if each relation
in S is 0-valid or each relation in $S$ is 1-valid, otherwise
the problems 1-FPN-SAT(S) 1-PN-SAT(S) are  PSPACE-complete.
\item
Let S be a finite set of finite arity Boolean relations such that
Rep(S) is the set of all weakly positive or Rep(S) is the set
of all weakly negative relations. Then
the problems 1-FPN(BC)-SAT(S) and 1-FPN(BC)-SAT$_c$(S) are PSPACE-complete
\end{enumerate}
\end{theorem}

\noindent
{\bf Proof:}
{\bf  Part 1:}
First, we give a polynomial time reduction 
from {\sf 1-FPN-3SAT} to {\sf 1-FPN-SAT$_c$(S)} as follows:
Since {\sf Rep(S)} is the set of all finite arity Boolean relations,
for each such $S$ and 3CNF clause $c$,
there is an existentially quantified (but not necessarily constant-free)
$S$-formula $f_c$ such that $c=[f_c]$.
Again up to an easy renaming of variables, there are only fourteen such
formulae.

Let $F = (U, C(i, i+1), m)$ be an instance of {\sf 1-FPN-3SAT}. 
Obtain a static  formula $F'$
by replacing every clause $c \in C(i, i+1)$ with the 
existentially quantified (but not necessarily constant-free)
{\sf S}-formula $f_c$ such that $c=[f_c]$. For each of these clauses,
let $f'_c$ be the {\sf S}-formula (again, possibly with
of constants) resulting from $f_{c}$ by deleting
all quantifiers after making sure that all quantified variables are
distinct from each other and from all free variables.  Without loss of
generality, we assume that, for all clauses $c$,
all variables of $f'_c$, that are not variables
of $f$, are local to $f'_c$.
We have now obtained an instance $F'$ of {\sf 1-FPN-SAT$_c$(S)}, 
which is satisfiable if and only if $F$ was satisfiable.

\noindent
Next, we now reduce {\sf 1-FPN-SAT$_c$(S)} to {\sf 1-FPN-SAT(S)}. 
We consider the four possible cases as discussed in Lemma \ref{le:sch4.3}.


\noindent
{\bf Cases (1) and (2):} 
In both these cases it is easy to see that {\sf 1-FPN-SAT(S)}  and
{\sf 1-PN-SAT(S)} are  polynomial time solvable.\\

\noindent
{\bf Case (3):} $[x]$ and $[\overline{x}]$ are contained in 
{\sf Rep$_{NC}$(S)}.\\
In this case, we modify $F'$ to obtain $F''$ as follows.
We introduce a new variable $x$ and replace all occurrences of
0 by $[x]$ and replace all occurrences of 1 by $[\overline{x}]$ and 
replace all existentially quantified variables by new variables. \\

\noindent
{\bf Case (4):} $[x \not\equiv y] \in$  {\sf Rep$_{NC}$(S)}.\\
In this case, each relation in $S$ is complementive.
We modify $F'$ as follows to obtain $F''$.
We introduce two new variables $x(t)$ and $y(t)$ and replace all occurrences of
0   by $x(t)$ and 
replace all occurrences of 1  by $y(t)$. We then add 
the $S$-formulas $[x_t \not\equiv y_t] \wedge [x_t \not \equiv r_{t}]
[r_t \not \equiv x_{t+1}] \wedge [y_t \not\equiv s_{t}]
[s_t \not\equiv y_{t+1}]$.  
to the static $S$-formula, and 
replacing all existentially quantified variables by new variables
$z_1(t), \ldots z_p(t)$.
Since $F'$ is complementive, $F''$ is also complementive.



Therefore, we have now obtained an instance $F''$ 
of {\sf 1-FPN-SAT(S)}, which is satisfiable
if and only if $F'$ was satisfiable. Clearly the hardness of
{\sf 1-FPN-SAT(S)} implies the hardness of the corresponding problems 
{\sf 1-PN-SAT(S)}.


\noindent
{\bf Part (2): }
We prove  this part for the problem {\sf 1-FPN(BC)-3SAT(S)}, when 
{\sf S} represents the set of all weakly negative relations.
The proof for {\sf 1-FPN(BC)-3SAT(S)} when {\sf S} represents 
the set of all weakly postive relations is similar and omitted.

We first prove the {\sf PSPACE}-hardness of {\sf 1-FPN-SAT$_c$(S)}
when {\sf Rep(S)} is the set of all weakly negative relations.
Since {\sf Rep(S)} is the set of all weakly negative  boolean relations,
for each such {\sf S} and {\sf 3CNF} clause $c$,
there is an existentially quantified (but not necessarily constant-free)
{\sf S}-formula $f_c$ such that $c=[f_c]$. 

Let $F$ be an instance of {\sf 1-FPN-3SATWN}. 
Obtain a {\sf 1-FPN}-specification $F'$ 
by replacing every clause $c \in C(i, i+1)$ with the 
existentially quantified (but not necessarily constant-free)
{\sf S}-formula $f_c$ such that $c=[f_c]$. For each of these clauses,
let $f'_c$ be the {\sf S}-formula (again, possibly with
constants) resulting from $f_{c}$ by deleting
all quantifiers after making sure that all quantified variables are
distinct from each other and from all free variables.  Without loss of
generality, we assume that, for all clauses $c$,
all variables of $f'_c$, that are not variables
of $f$, are local to $f'_c$.
Observe that the transformation is carried out by introducing temporary
variables $x_1(i), \ldots, x_r(i)$.
We have now obtained an instance $F'$ of {\sf 1-FPN-SAT$_c$(S)}, 
which is satisfiable if and only if $F$ was satisfiable. 

Next, we consider the case  of {\sf 1-FPNSAT(S)} when $Rep(S)$ 
represent all weakly positive relations. 
Clearly $[x]$ and $[\overline{x}]$  are in {\sf Rep(S)}.
This implies that 
\begin{enumerate}
\item
Each relation in {\sf S} is not 0-valid and

\item
Each relation in {\sf S} is not 1-valid.
\end{enumerate}
By Theorem \ref{le:sch4.3}, we have to consider two possible cases.

\noindent
{\em When $[x]$ and $[\overline{x}]$ is in {\sf Rep}$_{NC}$({\sf S}})

\noindent
{\em When $[x \not\equiv y ]$  is in {\sf Rep}$_{NC}$({\sf S}}).
The proof in both the cases is similar to proof in Part 1.\QED







\begin{theorem}\label{th:hsat_charcterize4}
Let S be a finite set of finite arity boolean relations. Let conditions
(a) through (e) be defined as follows.

\begin{description}
\item{(a)}
Every relation in S is 0-valid or every relation in $S$ is 1-valid.
\item{(b)}
Every relation in S is 1-weakly positive or
every relation in S is 1-weakly negative.
\item{(c)}
Every relation in S is weakly positive or
every relation in S is weakly negative.
\item{(d)}
Every relation in S is affine.
\item{(e)}
Every relation in S is bijunctive.
\end{description}

\begin{enumerate}
\item
If $S$ satisfies one of the conditions (b) (d) or (e)
then L-SAT$_c$(S) and 1-FPN(BC)-SAT$_c$(S) are  in P otherwise
L-SAT$_c$(S) 1-FPN(BC)-SAT$_c$(S) are  PSPACE-complete.

\item
If $S$ satisfies one of the conditions (a) (b) (d) or (e)
then L-SAT(S) and 1-FPN(BC)-SAT(S are  in P. Otherwise, L-SAT(S)and 
1-FPN(BC)-SAT(S  are PSPACE-complete.


\item
If {\bf S} satisfies conditions (a), (b), (c), (d) or  (e), then the problems  
{\sf 1-FPN-SAT(S)} and PN-SAT(S)  are in P. Otherwise, 
the problems  {\sf 1-FPN-SAT(S)} and PN-SAT(S)  are  PSPACE-complete.

\end{enumerate}

\end{theorem}




\noindent
{\bf Proof of Part 1.}\\
We first prove the theorem for the problems L-SAT$_c$(S).
By Theorem~\ref{th:sch3.0}, we have to consider five possible cases
depending on the type of relations in $S$. In the rest of the proof we
consider each of these cases in detail.

\noindent
{\em 1. Every relation in {\sf S} is weakly positive.}\\
By Theorem \ref{th:sch3.0} this implies that 
every relation in {\sf Rep(S)} is weakly positive. 
We need to consider two subcases in order to characterize weakly positive
relations:

{\em  1(a). Every relation in {\sf S} is 1-weakly positive.}\\
In this case, 
{\sf L-SAT$_c$(S)} is in {\sf P} by Theorem~\ref{th:h3sat1wneasy}.

{\em   1(b). Some relation in {\sf S} is not 1-weakly positive.}\\
In this case by Theorem~\ref{th:sch5.1}, 
{\sf Rep(S)} is the set of all weakly positive relations.
Hence, {\sf L-SAT$_c$(S)} 
is {\sf PSPACE}-complete by Theorems~\ref{th:fpn3satwnhard}
and \ref{th:translate}.

\noindent
{\em 2. Every relation in {\sf S} is weakly negative.}\\
Similar to Case 1.

\noindent
{\em 3. Every relation in {\sf S} is affine.}\\
By Theorem~\ref{th:sch3.0}, this implies that
every relation in {\sf Rep(S)} is affine. Hence, 
by Theorem~\ref{th:haffineeasy}
{\sf L-SAT$_c$(S)} is in {\sf P}. 

\noindent
{\em 4.  Every relation in {\sf S} is bijunctive.}\\
By Theorem~\ref{th:sch3.0}, if every relation in {\sf S} is bijunctive,
then every relation in {\sf Rep(S)} is bijunctive. Hence, {\sf L-SAT$_c$(S)} 
is in {\sf P}  by Theorem~\ref{th:h2satseasy}.

\noindent
{\em 5. {\sf Rep(S)} is the set of all logical relations.}\\
In this case {\sf L-SAT$_c$(S)} 
is {\sf PSPACE}-complete by Theorem~\ref{th:hsatshard}.


Next we prove the theorem for {\sf 1-FPN(BC)-SAT$_c$(S)}.
By our translation theorem, we know that whenever a {\sf L}-specified
problem is in {\sf P}, the corresponding {\sf 1-FPN(BC)}-specified problem 
is also in {\sf P}.
next, observe that the cases for which an L-specified problem has been
proved to be {\sf PSPACE}-complete, the proof actually consists of showing that
the corresponding problems are {\sf PSPACE}-complete for {\sf 1-FPN(BC)} 
specified instances. The proof now follows by these two facts.\\



\noindent
{\bf Proof of Part 2.}
We now prove the theorem for the problems {\sf L-SAT(S)} and
{\sf 1-FPN(BC)-SAT(S)}.  Since the problems
{\sf L-SAT(S)} are a subset of the problems {\sf L-SAT$_c$(S)} 
we get that whenever a problem {\sf L-SAT$_c$(S)} is  in 
P the corresponding problem {\sf L-SAT(S)} is also
in {\sf P}. Hence we need to consider three possible cases.

\noindent
{\em 1.  Every relation in {\sf S} is weakly positive and at least one relation
in {\sf S} is not weakly positive.}\\
In this case, by Theorems \ref{th:fpn3satwnhard} and \ref{th:translate}, 
we have that {\sf L-SAT(S)} is {\sf PSPACE}-complete.

\noindent
{\em 2. Every relation in {\sf S} is weakly negative and at least one relation
in {\sf S} is not weakly negative.}\\
In this case, by Theorems \ref{th:fpn3satwnhard} and \ref{th:translate}, 
we have that {\sf L-SAT(S)} is
{\sf PSPACE}-complete.

\noindent
{\em  3. {\sf Rep(S)}  is the set of all logical relations.}\\
By Theorem~\ref{th:hsatshard}, we have that if 
every relation in {\sf S} is 0-valid or every relation 
in {\sf S} is 1-valid then {\sf L-SAT(S)} is in {\sf P}
else the problem is {\sf PSPACE}-complete.



Next, consider {\sf 1-FPN(BC)-SAT(S)}. 
As in the proof of Part 1, we know that whenever a {\sf L}-specified
problem is in {\sf P}, 
the corresponding {\sf 1-FPN(BC)}-specified problem is also in {\sf P}.
next, observe that the cases for which an {\sf L}-specified problem has been
proved to be {\sf PSPACE}-complete, the proof actually consists of showing that
the corresponding problems are {\sf PSPACE}-complete for {\sf 1-FPN(BC)} 
specified instances. The proof of this part now follows by these two facts.\\


\noindent
{\bf Proof of Part 3.}
If {\sf S} satisfies conditions (a), (b), (d) or (e) then 
by the proof of Part 2 above, the problems
{\sf 1-FPN-SAT(S)} and {\sf 1-PN-SAT(S)} are in {\sf P}. 
Therefore we need to consider only the following cases to complete the proof.


\noindent
{\em 1. Every relation in {\sf S} is weakly positive or every relation in
{\sf S} is weakly negative}.\\
In this case by Theorems \ref{th:fpn3satwneasy} and \ref{th:pn3satwneasy}
the problems {\sf 1-FPN-3SATWP} , {\sf 1-FPN-3SATWN}, {\sf 1-PN-3SATWP}  and
{\sf 1-PN-3SATWN}  are in {\sf P}.

\noindent
{\em 2. When none of (a), (b), (c), (d) and (e) hold}\\
In this case we know that {\sf Rep(S)} is equal to the set
of all finite arity boolean relations. Since condition (a) does not hold,
by Theorem \ref{th:hsatshard}, we have that 
{\sf 1-FPN-SAT(S)} and {\sf 1-PN-SAT(S)} are {\sf PSPACE}-complete.

This completes the proof of the theroem.\QED
