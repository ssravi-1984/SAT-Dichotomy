\section{PSPACE-completeness of $\alpha$-3SAT}\label{sec:hard_3sat}

\begin{theorem}\label{th:fpn-3sat}
%%\begin{enumerate}
(1) 1-FPN-3SAT is in NSPACE$(n)$.~
(2) There is a $\theta(n)$ size 
quasi linear time reduction from the membership problem
for  non-deterministic linearly space bounded machine (LBA) 
to the problem 1-FPN-3SAT. 
Thus 1-FPN-3SAT is {\sf PSPACE}-complete. 
%%\end{enumerate}
\end{theorem}

\noindent
{\bf Proof:}  See supplementay material. 
\iffalse
%%%%%%%%%%%%%%%%%%%%%%%%%%%%%%%%%%%%%%%
{\em Part (1):}
We first show that the problem is in {\sf NSPACE($n$)}.
Observe that given a {\sf 1-FPN}-specification 
$\Gamma = (G(V,E), m)$, a Turing machine 
needs to maintain at any given time $t$ 
only assignments to variables at time $t$ and time $t+1$. 
Hence a non deterministic {\sf LBA}
can verify that the instance of {\sf 1-FPN-3SAT} is satisfiable as follows. At 
each step $t$ it guesses an assignment to the variables at grid point $t+1$.
It also remembers the assignment to the variables at grid point $t$. Using
these values it verifies that the clauses at time $t$ are indeed satisfied.
The number of time steps  $m$ can be kept track of using a counter of size
$O(m)$ (since $m$ is specified as a binary numeral). 
This proves that given an instance $\Gamma$ 
of {\sf 1-FPN-3SAT}, we can recognize in 
non-deterministic {\em linear } space (i.e in space $O(size(\Gamma))$) 
if the formula $G^m$ is satisfiable.

\noindent
{\em Part (2):}
Next, we prove {\sf 1-FPN-3SAT} is {\sf PSPACE}-hard. 
Given a non-deterministic {\sf LBA},
$M$, with input $x$ (where $|x| = n$), 
we construct an instance of {\sf 1-FPN-3SAT} $F(x)$ such that
$size(F(x)=  O(n)$,
and $x$ is accepted by $M$ if and only if $F(x,t,t+1)$ is satisfiable.
The reduction consists of two phases. 

\noindent
{\bf Phase 1:}
In the first phase, we
start with the given {\sf LBA} $M$ with input 
$x = (x_1, \ldots, x_n)$ and construct
a new {\sf LBA} $M_1$ which simulates $M$ on $x$ with the
following additional properties that
\begin{enumerate}
\item
if the {\sf LBA} $M$ does not accept $x$ then
each computation of $M_1$ on $x$ halts within $2^{c_0n}$ moves, and 
\item
if the {\sf LBA} $M$ accepts $x$ then $M_1$ has a cycling computation,
where the length of an ID never exceeds $O(|x|)$.

\end{enumerate}



$M_1$ can be constructed easily by adding an auxiliary  clock to serve 
as a counter. $M_1$ now just simulates $M$.  
If $M$ enters a final configuration, then $M_1$ repeats this
configuration. Its clear that 
$M_1$ accepts $x$
if and only if $M$ accepts $x$ and their number of
accepting computations is the same. Moreover it is easy to see that
$M_1$ has the two desired properties above.



\noindent
{\bf Phase 2:}
The second phase consists of constructing an instance 
$\Gamma = (F, M)$ of 
{\sf 1-FPN-3SAT} by a polynomial time reduction from $M_1$. 
Now we know that each ID of the Turing machine $M_1$ is 
of length $O(n)$, where $n$ is the size of the input.
Since $M_1$ is non-deterministic {\sf LBA}
we need to consider only $2^{Dn}$ different ID's for
our reduction. (Here $D$ is an appropriately chosen constant.)  
We can choose
an encoding of states and symbols of $M_1$ into words in $\{0,1\}^*$ so
that every ID of $M_1$ will consist of $c_1 \cdot n$ boolean variables where
$n=|x|$ and $c_1$ is a constant independent of $x$. Let $ID(t)$ denote
the {\sf ID} of the Turing machine at time $t$. We also have 
a set of $Dn+1$ boolean variables encoding a counter $\gamma(t)$. The counter
values range from $0$ to $2^{Dn}$. 
$F(x, t,t+1) = f_1(x,t,t+1) \wedge f_2(x,t,t+1) \wedge f_3(x, t, t+1)$. 
We discuss each of the three formulas $f_i(t,t+1)$, $1 \leq i \leq 3$.

\begin{enumerate}
\item 
The formula $f_1(x, t, t+1)$ encodes a counter which is given by 
\[\gamma(t + 1) =  (\gamma(t) + 1) \pmod{2^{Dn} +1}.\]
The intended meaning of the equation is  
that the counter resets to 0 after every $2^{Dn} +1$ time units.
It is easy to see that the counter 
can be simulated by a {\sf CNF} formula, in which each clause
has variables  that are no more than one time unit apart.
We briefly discuss how to simulate a counter here.  Let $q = Dn$
We use boolean
variables $d_q, d_{q-1}, \ldots d_0$ to simulate a counter.
The variable $d_0$ encodes the lowest order bit and the bit $d_q$ denotes the
highest order bit. We also use  boolean
variables $c_q, c_{q-1}, \ldots c_0$ to keep track of the carry bits 
required to do the addition. Let $d_i(t)$ denote
the copy of the variable $d_i$ at time unit $t$.
The set of narrow {\sf CNF} 
clauses needed to simulate the  counter  consist of the following.
The formula $f_1$ is expressed as follows:
\[f_1(x,t, t+1) \equiv (g_1 \Rightarrow h_1) \wedge (\overline{g_1} \Rightarrow h_2).\]

We describe each of the subformulas $g_1$, $h_1$ and $h_2$ in the 
following.
$g_1$ checks to see if the counter needs to be reset. If $g_1$ is true then
$h_1$ merely resets the counter; else $h_2$ increments it by one.
Hence $g_1$ is given by 
\[g_1  \equiv [(d_0(t) \wedge d_1(t) \wedge \ldots \wedge d_q(t) )\] 

The condition that counter resets to 0 after $2^{Dn}$ time units can now
be written using a CNF  formula $h_1$ as follows:
\[ h_1 \equiv (\overline{d_0(t+1)} \wedge \overline{d_1(t+1)} \wedge  
\ldots \wedge \overline{d_q(t+1)})].\]

As mentioned  earlier,
we have $q+1$ carry bits to do the addition. Let the carry bits
corresponding to the counter $\gamma(t)$ be $c_q, \ldots c_0$. Now the second 
part of the conjunct can be expressed as follows. $h_2$ is now defined
as a conjunction of the following clauses

\[\left(d_0(t+1) = \overline{d_0(t)} \right) \wedge
\bigwedge_{i= 1 }^{i = q} 
\left((d_i(t+1) = [\overline{d_i(t)} \wedge c_i(t+1)]+
               [d_i(t) \wedge \overline{c_i(t+1)}]) \right)\bigwedge \]
\[\left( c_0(t+1) = d_0(t) \right) \wedge  \bigwedge_{i =1}^{ i = q}  
\left(c_{i+1}(t+1) = d_i(t) \wedge c_i(t+1) \right) \]

Observe that we have $O(n)$ boolean variables encoding the counter.
Therefore, the size of each of the formulas $g_1, h_1$ and $h_2$ is linear
in the size of the input.
Furthermore each of the formulas $g_1$, $h_1$ and $h_2$ contain a linear
(in the number of boolean variables used to simulate the counter)  number
of clauses.  As a result, the implications 
$ (g_1 \Rightarrow h_1)$ and $(\overline{g_1} \Rightarrow h_2)$ can be written
in equivalent {\sf 3CNF} form using additional temporary variables.
The size of each {\sf 3CNF} formula is linear in original size of
the implications. 
Hence the size of $f_1(x,t,t+1)$ is also linear
in the size of the input. 


\item 
The formula $f_2(x, t, t+1)$ enforces the condition that when the
counter value is 0, the variables $X(t)$ encode the starting ID of the
Turing machine.
Here, $start(ID(t))$ denotes a  {\sf CNF} 
formula which checks if $ID(t)$ is the initial ID of the machine.
Thus $f_2(t, t+1)$  is a {\sf 3CNF} formula which
encodes the  implication $(\gamma(t) =0) \Rightarrow start(ID(t))$.\\ 
Again, we can verify that $f_2(x, t, t+1)$ 
can be written as a {\sf 3CNF} formula in polynomial time. Again by standard
techniques it follows that the formula $f_2(x,t, t+1)$ is of size $O(n)$. 



\item
The formula $f_3(x, t, t+1)$ is needed to ensure that starting at 
the second ID, each subsequent ID
of $M_1$ follows from the previous ID by using the
transition function of $M_1$.
(Recall that the notation $(X {\vdash}_M^j Y)$ means that machine $M$, 
starting with ID $X$, can produce the ID $Y$ in exactly $j$ steps.) 
Thus $f_3(t)$ is a {\sf 3CNF} formula encoding the following implication. 
\[(1 \leq \gamma(t) \leq 2^{Dn}) \Rightarrow (ID(t-1) {\vdash}_M ID(t)) \]
The function ($ID(t) {\vdash}_M ID(t+1)$) can be expressed
by {\sf 3CNF} formula whose sizes are linear in $n$ as shown in \cite{Hu73a}.
Moreover the {\sf 3CNF} formula depends on the current value of the counter. 
Hence the {\sf CNF} formula $f_3(x, t, t+1)$ is  
narrow periodic formula, since the clauses  at time $t$  would contain
variables only from times $t$ and $t+1$.


\end{enumerate}

\noindent
The expanded finite periodic {\sf 3SAT} 
instance is $\bigwedge_{t =0}^N F(x,t, t+1)$,
where $N = 2^{2Dn}$.

We now prove the correctness of our reduction.
If the Turing machine $M$ accepts $x$ then we know that 
$M_1$ has a cycling computation. 
Hence by setting $d_t =0$, we can  ensure that $f_2(0)$ is satisfied.
The consistency condition now forces the formula to be satisfiable 
Conversely, assume that the formula is satisfiable. 
Since $D$ ( and hence $N$) are 
suitably large integers, it is  guaranteed that the  simulation
must be carried out for enough steps so that the Turing machine $M_1$
goes through the sequence $d_t =0, d_t = 1, d_t = 2, \cdots d_t = 2^{Dn}$. 
This implies that  the formulas $f_2(t)$ and $f_3(t)$
would be true from then on 
and therefore the  Turing machine $M$ accepts $x$.\QED
%%%%%%%%%%%%%%%%%%%%%%%%%%%%%%%%%%%%%%%
\fi

\begin{corollary}\label{th:l3sathard}
%%\begin{enumerate}
(1) 
The problem 1-FPN(BC)-3SAT is in NSPACE(n).
There is a quasi-linear
time linear size reduction from the membership problem for non-deterministic
LBA to 1-FPN(BC)-3SAT.
Thus the  problem 1-FPN(BC)-3SAT is PSPACE-complete.~
(2)
The problem L-3SAT is in DSPACE(n).
There is a square time
square size reduction from the membership problem for non-deterministic
LBA to L-3SAT. Thus the  problem L-3SAT is PSPACE-complete 
%\end{enumerate}
\end{corollary}

\noindent
{\bf Proof:} See supplementary material. 

\iffalse
%%%%%%%%%%%%%%%%%%%%%%%%%%%
The {\sf PSPACE}-hardness of {\sf 1-FPN-3SAT(BC)} follows 
immediately from Theorem \ref{th:fpn-3sat}.
By arguments similar to those
presented in the proof of Theorem \ref{th:fpn-3sat}, 
the problem {\sf 1-FPN(BC)-3SAT}  is seen to be in {\sf NSPACE(n)}.


The {\sf PSPACE}-hardness of  the problem {\sf L-3SAT} follows
from Theorem \ref{th:fpn-3sat} and the Translation theorem. Specifically,
there is a $O(n^2)$ size reduction from the membership problem for a  
non-deterministic {\sf LBA} to the problem  {\sf L-3SAT}.
We now show that {\sf L-3SAT} is solvable in {\sf DSPACE(n)}
by using backtracking. 
Let $\displaystyle{F_i(X^i)=(\bigwedge_{1 \leq j \leq  l_i} 
F_{i_j}(X^i_j,Z^i_j)) \bigwedge f_i(X^i,Z^i)}$.
Here $0 \leq i_j <i$.
Given an assignment of $X^i$, 
we evaluate $F_i(X^i)$ as follows:
For each assignment of $Z^i$:
For $1 \leq j \leq  l_i$, 
check if $F_{i_j}(X^i_j,Z^i_j)$ is true and store the values of 
$F_{i_j}(X^i_j,Z^i_j)$.
These values require space linear in $l_i$. Finally evaluate $f_i(X^i,Z^i)$.
This evaluation can be carried out in deterministic space linear in the size of
$f_i$ since it is a {\sf CNF} formula.
If no such assignment to $Z^i$ is found, 
$F_i(X^i)$ is false. Otherwise, $F_i(X^i)$ is true.
Thus, if $F_k$ for $k < i$, can be solved in space linear in their
representation, $F_i$ can be solved in space linear in its representation,
since the space needed to $F_i$ in the space to store the values of
$X^i$, $Z^i$ and the intermediate results $F_{i_j}(X^i_j,Z^i_j)$. This
allows us to do the backtracking in {\sf DSPACE(n)}.
$F_1$ can be solved in {\sf DSPACE(n)}  since it is a {\sf CNF} formula.
Hence by induction on the number of levels in the hierarchy, membership of
the problem {\sf L-3SAT} in {\sf DSPACE(n)} follows.\QED
%%%%%%%%%%%%%%%%%%%%%%%%%%%
\fi

The results yield a tight lower and upper bound (under certain complexity 
theoretic assumptions)
on the space required for solving the problem {\sf L-3SAT}. To see this, 
observe that if the transformation from {\sf 1-FPN}-specifications to
{\sf L}-specifications yielded instances smaller than $O(n^2)$ size,  or
if {\sf L-3SAT} could be solved in time less than {\sf DSPACE(n)}, 
then it would immediately imply that {\sf NSPACE(n)} can be simulated  in 
deterministic space less than $O(n^2)$. 
This would improve Savitch's \cite{Sa70} 
result concerning simulating  non-deterministic space bounded 
Turning machines by deterministic space bounded Turning machines.








\begin{corollary}\label{th:pn3sat}
The problem 1-PN-3SAT is in NSPACE(n).
There is a quasi-linear
time linear size reduction from the membership problem for non-deterministic
LBA to 1-FPN(BC)-3SAT.
Thus the  problem 1-PN-3SAT is PSPACE-complete. 
\end{corollary}

\noindent
{\bf Proof:}
Orlin \cite{Or82a} and Papadimitriou \cite{Pa94} have shown that
the problem {\sf 1-PN-3SAT} is in {\sf NSPACE(n)}. 
We only prove that the problem is {\sf PSPACE}-hard by a linear size
reduction from the membership problem for a non-deterministic {\sf LBA}.
(In contrast, The reduction in \cite{Or82a} is a square size reduction.)  
Observe that if an instance of {\sf 1-PN-3SAT} has a solution
then the assignment to the variables is periodic. If there are $n$ variables
in the static formula we can have no more than $2^{2n}$ distinct assignments to
the variables to consider. 
Hence starting from  an instance of {\sf 1-FPN-3SAT} 
$(F(U,C(i, i+1)),m)$ where $n = |U|$   and $m = 2^{2n}$ we  see that 
$F^m$ is satisfiable if and only if $F^{\infty}$ is satisfiable.
This completes the proof that the problem {\sf 1-PN-3SAT} is 
{\sf PSPACE}-complete.\QED


\noindent
{\bf Remark 1:}
The basic technique underlying the proof of {\sf PSPACE}-hardness of 
{\sf 1-FPN-3SAT}
is the ability to use a counter to implicitly make sure that there is a
sequence of formulas which check if the TM starts right. 



\noindent
{\bf Remark 2:} 
The above results show the basic ideas underlying the 
{\sf PSPACE}-hardness results in both Lengauer and Wagner \cite{LW92} 
and in Orlin 
\cite{Or82a}. Specifically, our results show that a very simple form 
repetitive structure that can be represented  
by {\sf 1-PN} as well as the {\sf L}-specification 
makes the problems {\sf PSPACE}-hard.  







As our next corollary shows, {\sf 1-FPN-3SAT} is {\sf PSPACE}-hard 
even when restricted to formulas with bandwidth $O(\log {\cal N})$. 
Recall that ${\cal N}$ denotes the size of the encoding of the expanded formula
using standard encodings.

\begin{corollary}
There is a quasi-linear
time linear size reduction from the membership problem for non-deterministic
LBA to 1-FPN-3SAT even when restricted to formulas with bandwidth
$O(\log {\cal N})$.
Thus the  problem 1-FPN-3SAT is PSPACE-complete even when restricted
to formulas with bandwidth $O(\log {\cal N})$. 
\end{corollary}

\noindent
{\bf Proof:}
The proof follows by observing that the following numbering
scheme yields a $O(\log {\cal N})$ bandwidth layout. 
Let $C(i, i+1)$ have $p$ clauses.
Then number the clauses at time $t$ using numbers from $pt$ to $(p+1)t$.
The numbering of clauses at a given time $t$ can be carried out in
any arbitrary order.\QED


\noindent
{\bf Remark 3:} The above corollary in conjunction with local replacement
type reductions between problems specified using standard specifications
can be used to prove that several classical graph problems are 
{\sf PSPACE}-hard even for $O(\log {\cal N})$ bandwidth bounded graphs
specified using {\em either}  {\sf 1-FPN}-specifications or
strongly 1-level-restricted {\sf L}-specifications (since the 
{\sf L}-specification obtained by the translation theorem represents an
isomorphic graph (formula)). 



Next, we discuss the {\sf PSPACE}-hardness of the problems 
{\sf 1-FPN(BC)-3SATWP}, {\sf 1-FPN(BC)-3SATWP} and 
{\sf 1-FPN(BC)-3SAT(S)}, such that every relation in {\sf S} is 
weakly positive or every relation in {\sf S} is weakly negative.
Recall that, a logical relation $R$ is {\bf weakly negative} if
$R(x_1,x_2,\ldots)$ is logically equivalent to some CNF formula having at
most one unnegated variable in each conjunct. Also recall that,
{\sf 1-FPN-3SATWP} (problem 3SATWP specified using 
{\sf 1-FPN}-specifications)  is the problem of determining if a 3CNF formula
$F^{M}(U^{M}, C^{M})$ which contains at most one negated 
literal per clause and is specified by $\Gamma= (U,C(i),M)$ is satisfiable.



\begin{theorem}\label{th:fpn3satwnhard}
There is a quasi-linear
time linear size reduction from the membership problem for deterministic
LBA to problems 1-FPN(BC)-3SATWN and  1-FPN(BC)-3SATWP.
Thus the problems 1-FPN(BC)-3SATWN and  1-FPN(BC)-3SATWP PSPACE-complete
\end{theorem}



\noindent
{\bf Proof:} \\
Part (1):
We prove  this part for the problem {\sf 1-FPN(BC)-3SATWN}. 
The proof for {\sf 1-FPN(BC)-3SATWP} is similar.


The membership of the problem {\sf 1-FPN(BC)-3SATWN} in {\sf NSPACE(n)} 
follows since the problem {\sf 1-FPN(BC)-3SAT} is in {\sf NSPACE(n)}.  
To prove that {\sf 1-FPN(BC)-3SATWN} is {\sf PSPACE}-hard we give a linear size
reduction from the acceptance problem of a {\bf deterministic} {\sf LBA}.
Let $ID(t)$ denote the  instantaneous description of a
a deterministic {\sf LBA} $M$ at time time $t$.
Given $M$ and input $x$ such that $|x| = n$ we create
an instance $(F(x,t,t+1),m)$ of {\sf 1-FPN(BC)-3SATWN}, such that 
$\wedge_{t = 0}^{t = N} F(x, t, t+1)$ is 
satisfiable if and only if $M$ accepts $x$. 
The reduction is similar to one presented for  {\sf 1-FPN-3SAT}. 
The formulas encoding $(ID(t) \vdash ID(t+1))$, and $start(ID(t))$
can  be represented by a weakly negative formula
as in the {\sf P}-hardness of {\sf UNIT} given in \cite{JL77}. 
By negating all literals, we can 
obtain a weakly positive formula proving the {\sf PSPACE}-hardness of 
{\sf L-3SATWP}. Let $a \in \{ \# \} \cup T \cup  (S \times T)$, where 
$T$ denotes the tape symbols and $S$ denotes the set of states. 
Let $P^a_i(t)$ be a boolean variable which means that the contents
of $i^{th}$ tape cell at time $t$ is $a$. Observe that for a given value of $t$
the number of variables $P^a_i(t)$ is $O(n)$. 
The formula $g(x,0)$ is now represented as
\[g(x, 0) \equiv 
\left( P_{1}^{(q_0,a_1)}(0) \wedge P_{2}^{a_2}(0),\ldots \wedge 
P_{n}^{a_n}(0) \right) \bigwedge_{i} \left(\bigwedge_{a \neq b} 
(~\overline{P_{i}^a(0)} \vee  ~\overline{P_{i}^b(0)}) \right)\]
The first part represents the condition that the first ID corresponds to the
input and the second part represents the condition that one position
cannot contain two distinct symbols. Hence, the above formula represents the
condition that the first $ID$ is correct.  Also observe that since the number 
of tape symbols are constant, the size of $g(x,0)$ is linear in the size of 
the input.

Next, we represent  $(ID(t) {\vdash}_M ID(t+1))$
as follows:
Let $f:(T \cup (S \times T))^3 \rightarrow T \cup (S \times T)$ be the finite
function such that if positions $i-1$, $i$ and $i+1$ of the $ID(t)$ 
contain $a$, $b$ and $c$ respectively, then position $i$ of the $ID(t+1)$ 
must contain $f(a,b,c)$. The determinism of $M$ ensures that
$f$ is single valued. We express the requirement that $ID(t+1)$ 
is appropriately  determined by $ID(t)$ as follows:


\[g(x, t,t+1) \equiv  \bigwedge_{i =1}^{ i = n} ~~~ \bigwedge_{a,b,c \in T} 
\left(\left(~P_{i-1}^{a}(t) \wedge P_{i}^{b}(t) \wedge 
P_{i+1}^{c}(t) \right) \Rightarrow ~P_{i}^{f(a,b,c)}(t+1) \right) \]
which can be written as an equivalent weakly negative formula as follows:

\[g(x, t,t+1) \equiv  \bigwedge_{i =1}^{ i = n} ~~~ \bigwedge_{a,b,c \in T} 
\left(~\overline{P_{i-1}^{a}(t)} \vee \overline{P_{i}^{b}(t)}
\vee \overline{P_{i+1}^{c}(t)}  \vee ~P_{i}^{f(a,b,c)}(t+1) \right) \]

Now, each four literal clauses replaced by equivalent three literals clauses
by adding new auxiliary variables $T^{f(a,b,c)}_{i}(t)$. Therefore
$g(x, t,t+1)$  is rewritten as 

\[ \bigwedge_{i =1}^{ i = n} ~~~ \bigwedge_{a,b,c \in T} 
\left(
\left(~\overline{P_{i-1}^{a}(t)} \vee \overline{P_{i}^{b}(t)}
\vee T_{i}^{f(a,b,c)}(t) \right)
\wedge 
\left( \overline{T_{i}^{f(a,b,c)}(t)} \vee
\overline{P_{i+1}^{c}(t)}  \vee ~P_{i}^{f(a,b,c)}(t+1) \right)\right) \]


$F(x,t,t+1) = g(x, 0) \cup g(x,t, t+1)$. Let $N = 2^{2Dn}$.
The instance output by the reduction is $(F(x,t, t+1), N)$
The corresponding expanded formula is given as 
\[ F^N(x) = g(x, 0) \wedge \bigwedge_{t = 0}^{N}g(x, t, t+1). \] 
Again observe that since the number of tape symbols are constant and $i$ 
varies  from $1$ to $n$,  
the size of the formula $g(x, t, t+1)$ is linear in the size of the input.
Also, observe that representation of $N$ is of size $O(n)$. Therefore
the size of the instance obtained as a result of the reduction is $O(n)$.
Here, 
$N$ is suitably chosen large integer so that the simulation can be carried 
out for enough number of steps. This completes the proof of the 
first part.\QED



\noindent
{\bf Remark 5:}
The above result serves as a useful starting point for proving 
{\sf PSPACE}-hardness of several {\sf L}-specified 
and {\sf 1-FPN(BC)}-specified 
problems whose corresponding non-succinct versions are $\log$-complete
for {\sf P}. For example, we get a direct proof of the {\sf PSPACE}-hardness
of {\sf L-MCVP} and {\sf 1-FPN(BC)-MCVP}; 
the circuit value problem for {\sf L}-specified and {\sf 1-FPN(BC)}-specified
circuits.


Although, we have a linear size reduction from the membership problem for
a deterministic {\sf LBA}, we do not know how to solve the problem in 
{\sf DSPACE(n)}.
