\section{Introduction}
\label{sec:intro}

Large objects with a highly repetitive structure can often be represented
succinctly.
In the past, several methods have been proposed to succinctly 
represent such objects.
Here we consider two of these methods, namely
(i) hierarchical specifications  
\cite{Ga82,GW83,LW92,BOW83,RH93,LW93,MH+93a,MH+94,marathe1995complexity,Wa86},
and (ii) periodic specifications 
\cite{CM91,HW92,IS87,KO91,KS88,Or82a,Wa93}.
 


Hierarchical specifications are useful in describing large scale systems with
very regular structures. 
Using hierarchical specifications, the overall design of an object can be 
partitioned into the design of a collection of modules;
this is a 
much more manageable task compared to producing a complete design in one step.
Such a top down (or hierarchical design) approach 
also facilitates the development of computer aided design (CAD) systems, since
low-level objects can be  incorporated into libraries and can thus be made
available as submodules to designers of large scale  objects.
Other areas where hierarchical specifications have found applications are
VLSI layout \cite{Le82,Le86,Le90,HLW92,HW92,RH93},  
finite element analysis, software engineering and Datalog queries 
(see \cite{HLW92,Ma94,marathe1995complexity,bottcher2020simulation,goller2005fixpoint,brenguier2012comparison,lohrey2012model,tapken1999implementing,AC94} 
and the references therein).  Recently, there has been interest in
inferring the inherent hierarchy that graphs might exhibit from the
standpoint of graph neural networks \cite{ying2018hierarchical}.

Periodic specifications
can also be used to define large scale systems with highly regular structures.
Using  periodic specifications, 
large objects are  described as repetitive connections of a basic module.
Frequently, the modules are connected in a straight line 
but the basic modules can also  be repeated in two or higher
dimensional patterns. Periodic specifications are also used to 
model time variant  problems,  where the constraints
or demands  for any one period is the same as those for preceding or 
succeeding periods. 
Periodic specifications 
have applications in such diverse areas  as  transportation planning
\cite{Or82a,HLW92,marathe1995complexity,drucker2019cyclic,ho2015cyclic},
parallel programming \cite{HLW92,KMW67} wireless networks 
\cite{andreou2002radiocoloring,fotakis2006radiocolorings} and 
VLSI design \cite{IS87,IS88}. 
See \cite{chen2003periodic,chen2005periodic} for a good discussion and 
references on periodic constraint satisfaction problems.

Orlin has studied periodically specified problems over an infinite 
horizon; i.e the objects considered are infinite \cite{Or82a}. 
Ford and Fulkerson \cite{FF58} and Wanke \cite{Wa93} have 
studied the finite horizon
versions of periodic problems and Gale \cite{Ga59} has studied problems for
periodic specifications with fixed starting points. All the above mentioned
studies are for 1-dimensional periodic specifications.
Other 
researchers have studied 2-dimensional and more generally $d$-dimensional
periodic specifications; 
see \cite{drucker2019cyclic,ho2015cyclic,CM91,IS87,KO91,KS88,Wa93}. 

An important feature of  both hierarchical and periodic 
specifications is that they
can be much  more concise in describing  objects than standard  specifications.
In general, the size of an object can be \emph{exponential} in the
size of its periodic or hierarchical specifications.
Since, the  complexity of solving a problem 
is usually measured as a function of the size of the specification of the 
problem instance,  
the complexity of a problem when instances are specified using
standard descriptions can
be  different than the complexity of the same problem 
when instances are specified hierarchically or periodically.
For example, while the {\sf 3-COLORING} problem is 
{\sf NP-}complete when the graphs are represented by adjacency matrices
or by adjacency lists \cite{GJ79}, it is 
{\sf PSPACE}-complete when instances are specified
hierarchically or periodically \cite{LW92,Or84b}.
On the other hand, the {\sf 2-COLORING} problem is solvable in polynomial time
{\em even} when instances are specified by the hierarchical
specifications of Lengauer et al. \cite{LW92,LW87a} or by  the 
1-dimensional periodic specifications of Orlin \cite{Or82a}.




In this paper, we study the complexity of combinatorial, graph and 
generalized {\sf CNF} satisfiability problems,
when instances are specified in any of the following ways:

\begin{enumerate}
\item
the 1-dimensional finite periodic narrow specifications (denoted by
{\sf 1-FPN}-specifications) of Ford and Fulkerson \cite{FF58}
and of Wanke \cite{Wa93}; 
\item
the 1-dimensional finite periodic narrow specifications with explicit 
boundary conditions (denoted by {\sf 1-FPN(BC)}-specifications)
of Gale \cite{Ga59} and others; 
\item
the 2-way infinite 1-dimensional narrow
periodic (sometimes called dynamic) specifications  (denoted by
{\sf 1-PN}-specifications) 
of Karp et al. and Orlin \cite{KMW67,Or82a}; and 
\item
the hierarchical specifications (denoted by {\sf L}-specifications) of 
Lengauer \cite{LW87a}. 
\end{enumerate}

\noindent
Let $\Pi$ be a problem, whose instances are specified {\em non-succinctly} 
using one of the standard specifications in the literature. For example,
instances of {\sf CNF} satisfiability problems can be specified non-succinctly
by {\sf CNF} formulas and by sets of clauses, with each clause being a set of 
literals.
Here, we use {\sf 1-FPN-}, {\sf 1-FPN(BC)-}, {\sf 1-PN-} and {\sf L-}$\Pi$ to
denote the problem $\Pi$ 
when its instances are specified succinctly by {\sf 1-FPN-}, 
{\sf 1-FPN(BC)-}, {\sf 1-PN-} and {\sf L-} specifications, respectively. 
We use $\alpha$-$\Pi$ to denote the problems {\sf 1-FPN-}, {\sf 1-FPN(BC)-},
 {\sf 1-PN-} and {\sf L-}$\Pi$; and we use {\em succinct} specification 
to mean {\sf 1-FPN-}, {\sf 1-FPN(BC)-}, {\sf 1-PN-} and {\sf L-}specification. 
Thus for example, {\sf 1-FPN-3SAT} denotes the problem {\sf 3SAT}
when instances are specified  by {\sf 1-FPN-}specifications; and 
$\alpha$-{\sf 3SAT} denotes the problems {\sf 1-FPN-}, {\sf 1-FPN(BC)-}, 
{\sf 1-PN-} and {\sf L-3SAT}.

\smallskip

Our results are summarized in the next section.
