\section{Summary of Results}
\label{sec:summary}

The rest of the paper is organized as follows.  In Subsections 
\ref{sec:trans} through \ref{sec:comp} 
we summarize our  results and discuss their relationship with related 
results in the literature. Section \ref{sec:def} gives both the 
definitions and examples of the kinds of  succinct 
specifications considered.
In Section \ref{sec:translation}
we  prove a translation theorem  relating the 
{\sf 1-FPN-} and {\sf 1-FPN(BC)-}specifications 
to equivalent {\sf L-}specifications. 
 In Section \ref{sec:hard_3sat},
we prove the {\sf PSPACE}-hardness of the problems
\begin{quote}
 {\sf 1-FPN-3SAT}, {\sf 1-FPN(BC)-3SATWP},  {\sf L-3SAT}, and
{\sf L-3SATWP}.
\end{quote}
In Section \ref{sec:poly} we discuss several kinds of  
{\sf L-} and {\sf 1-FPN-}specified generalized {\sf CNF} 
 satisfiability problems that are solvable in 
polynomial time. In Section \ref{sec:char} assuming 
{\sf P}$\neq${\sf PSPACE}, we complete the 
characterization of the polynomial time solvability
of the problems $\alpha-${\sf SAT(S)}. 
%%%%
%%%% The current version does not include proofs of the results mentioned below.
%%%%
%% In  Section \ref{sec:applications}, 
%% we outline how efficient reductions involving local replacement \cite{GJ79}
%% of the problems {\sf 3SAT, Planar 3SAT, 3SATWP,} etc., can be extended 
 %% to efficient reductions of the corresponding problems 
%% $\alpha$-{\sf 3SAT}, $\alpha$-{\sf Pl-3SAT}, $\alpha$-{\sf 3SATWP}, etc. 
%% This enables us to prove  the {\sf PSPACE-}hardness of a number of 
%% {\sf P-} and {\sf NP-}hard problems in the literature when instances are
%% specified succinctly, including the various {\sf PSPACE-}hard problems 
%% for {\sf 1-PN-}specifications in \cite{Or82a} and for {\sf L-}specifications
%% in \cite{LW92}. 

 
\vspace*{-0.15in}
\subsection{Translation Theorem}\label{sec:trans}
In Theorem 4.1(called the {\bf Translation Theorem}),
we prove that a {\sf 1-FPN-} or {\sf 1-FPN(BC)-} specification of a graph
or  formula can be translated in polynomial time into an
{\sf L}-specification of an isomorphic  graph or formula, respectively.
This theorem implies, for many problems
$\Pi$ including all problems considered here,  the problems 
{\sf 1-FPN-} or {\sf 1-FPN(BC)-}$\Pi$ are 
polynomial time reducible to  the problem {\sf L-}$\Pi$.

\vspace*{-0.15in}
\subsection{Complexity of Boolean Constraint Satisfaction Problems}\label{sec:sat}
In Sections \ref{sec:hard_3sat} through \ref{sec:char}, we study the complexities 
of the 
 generalized {\sf CNF}  satisfiability problems {\sf SAT(S)} of \cite{Sc78},
when instances are specified succinctly.
Assuming {\sf P} $\neq$ {\sf PSPACE}, we characterize completely 
the polynomial time solvability of the problems  $\alpha$-{\sf SAT(S)} as follows. We give sufficient conditions on a finite set {\sf S} of finite-arity 
Boolean relations for the problem  $\alpha$-{\sf SAT(S)} to be 
{\sf PSPACE-}complete. We also show, for all such {\sf S} {\em not}
 satisfying 
these conditions, that the problem $\alpha$-{\sf SAT(S)} is polynomial 
time solvable. In Section 6 as steps in obtaining this characterization,
we prove that the problems {\sf 1-FPN-3SAT}, {\sf 1-FPN(BC)-3SATWP}, and 
{\sf 1-FPN(BC)-3SATWN} are {\sf PSPACE-}hard. By the Translation
Theorem, this implies that the problems {\sf L-3SAT}, {\sf L-3SATWP},
and {\sf L-3SATWN} are also {\sf PSPACE-}hard.

\vspace*{-0.15in}
\subsection{A Step Towards a Complexity Theory for Succinctly Specified 
Problems}
In Section 8, we outline how a polynomial time reduction involving
local replacement \cite{GJ79} from the problem 
{\sf 3SAT, NAE 3SAT, Pl-3SAT,} etc., to a problem $\Pi$ can be extended
to obtain a polynomial time reduction from the corresponding problem 
$\alpha$-{\sf 3SAT}, $\alpha$-{\sf NAE 3SAT} ,
 $\alpha$-{\sf Pl-3SAT}, etc., to the problems
 $\alpha$-$\Pi$. Together with the results outlined in 
Sections 2.1 and 2.2, this enables us to do the following:
\begin{enumerate}
\item derive alternative and unified proofs of the {\sf PSPACE-}hardness 
results for {\sf 1-PN-} specifications in \cite{Or82a} and for
 {\sf L-}specifications in \cite{LW92}; and 
\item derive several new {\sf PSPACE-}hardness results for succinctly 
specified problems. 
\end{enumerate}

Often, our {\sf PSPACE-}hardness  proofs involve 
reductions that are more time and/or space efficient than those in 
\cite{Or82a,LW92}. Many of these hardness results hold, even when 
restricted to specifications of $O(\log {\cal N})$ 
bandwidth-bounded planar graphs.
 



\vspace*{-0.15in}
\subsection{Comparison with Related Work}\label{sec:comp}
Orlin \cite{Or82a} proved that the problem {\sf 1-PN-3SAT} is 
{\sf PSPACE-}complete. He used this and known reductions from {\sf 3SAT} 
to prove the {\sf PSPACE-}hardness of the problems 
\begin{quote}
{\sf 1-PN-KNAPSACK, 1-PN-HAMILTON-PATH(1-PN-HP), 1-PN-3COLORING,} and 
{\sf 1-PN-3DM}.
\end{quote}
Wanke \cite{Wa93} has also proved {\sf PSPACE-}hardness results for periodically-specified problems; but his results do not hold for 1-dimensional 
periodically-specified problems. Lengauer and Wagner \cite{LW92} have proved
the {\sf PSPACE}-hardness of the problems 
\begin{quote}
{\sf L-3COLORING, L-INDEPENDENT SET(L-IS), L-HAMILTON CIRCUIT(L-HC), L-MONOTONE CIRCUIT  VALUE PROBLEM(L-MCVP), L-NETWORK FLOW(L-NF),} and {\sf L-ALTERNATING GRAPH ACCESSIBILITY PROBLEM(LAGAP).}
\end{quote}
Their {\sf PSPACE-}hardness results for the problems {\sf L-3COLORING, L-IS,}
and {\sf L-HC} hold for $O(\log {\cal N})$ bandwidth-bounded instances. Several
{\sf PSPACE-}hardness results for {\sf L-}specified unit disk graphs were 
presented in our paper \cite{MR+93}. 
Our results extend the results above from \cite{Or82a,LW92}  
in the following ways:

%%\begin{enumerate}
%%\item 
\noindent
1.~ Previously, the complexities of periodically- and hierarchically-specified
problems have been studied separately. Our results show that there is a
 close correspondence between Orlin's {\sf PSPACE-}hardness results for {\sf 
1-PN-}specified problems and Lengauer and Wagner's {\sf PSPACE-}hardness 
results for {\sf L-}specifies problems.

\smallskip
\noindent
2.~
The only previous work on the complexities of the problems 
$\alpha$-{\sf SAT(S)} is that of Orlin \cite{Or82a} on the 
{\sf PSPACE-}hardness of the problem {\sf 1-PN-3SAT}. (Several references
to the results in this paper occur in our paper \cite{MR+93}.)

\smallskip
\noindent
3.~
No {\sf PSPACE-}hardness results, for succinctly presented 
planar problems, have been presented previously. No {\sf PSPACE-}hardness 
results of any kind have been presented previously, for problems specified
succinctly by either {\sf 1-FPN-} or {\sf 1-FPN(BC)-}specifications.

\smallskip
\noindent
4.~
Our {\sf PSPACE-}hardness results for {\sf L-}specified
problems imply the {\sf PSPACE-}hardness of the problems 
{\sf L-MCVP} and {\sf L-AGAP} for $O(\log {\cal N})$ 
bandwidth-bounded instances.

\smallskip
\noindent
5.~
Several of our  reductions  used to prove {\sf PSPACE-}hardness
are more  efficient in both   time  and space than  the  corresponding
reductions in \cite{Or82a,LW92}. For example,   our proof of the  {\sf
PSPACE-}hardness  of  {\sf  1-PN-3SAT}  is   by an  $O(n\log n)$  time
reduction of the acceptance problem  for a nondeterministic {\sf LBA}.
That of Orlin  \cite{Or82a} is  by  an $\Omega(n^2 \log n)$  time  and
space reduction  of the same  problem. Under the  plausible assumption
that  there exist   {\sf  LBA}s   whose acceptance  problems   require
$2^{\Omega(n)}$ time  on deterministic Turing machines,  our reduction
implies    that     the    problem     {\sf    1-PN-3SAT}     requires
$2^{\Omega(\frac{n}{\log    n})}$   time  on   deterministic    Turing
machines. That of Orlin only  implies that the problem {\sf 1-PN-3SAT}
requires   $2^{\Omega(\frac{\sqrt{n}}{\log n})}$ time on deterministic
Turing machines.

\smallskip
\noindent
6.~
Our proofs show that it is possible to use knowledge 
about {\em local reductions} from generalized {\sf CNF} satisfiability problems
in the standard case to predict the complexity of problems, when instances
are succinctly specified. 

%% Commented out as the discussion is not in Section 8.
%% We discuss points 5 and 6 in greater detail in Section 8.
%\end{enumerate}
