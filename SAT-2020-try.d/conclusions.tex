\section{Conclusions}\label{sec:conclusion}

We proved a translation theorem which allowed us to relate the
{\sf L-} and {\sf 1-FPN-}specification languages. 
The main appeal of the translation
theorem was in the corollaries; namely the various polynomial time
results and {\sf PSPACE}-hardness results for {\sf L}- and 
{\sf 1-FPN}-specified problems
We then investigated the complexity of an infinite collection of 
satisfiability problems SAT(S) when instances are specified by 
{\sf L-, 1-FPN, 1-FPN(BC)} or {\sf PN}-specification. 
We completely characterized the
complexity of the problems $\alpha$-{\sf SAT(S)}. In particular, 
we  showed that either the problems belong to
{\sf P} or they are {\sf PSPACE}-complete. 
Moreover, in conjunction with our translation theorem and known local 
reduction they provide alternative and unified {\sf PSPACE}-hardness proofs for
several {\sf L}- or 
{\sf PN}-specified problems considered in \cite{LW92,Or82a}.
We also prove the {\sf PSPACE}-hardness of several other 
{\sf L-, 1-FPN, 1-FPN(BC)} or {\sf PN}-specified problems.  
%Table 1 contains example of the results obtained in this paper.



Another feature of the results is to point out the relative 
economy of descriptions of
the variants of 1-dimensional narrow periodic specifications. For instance our
results show that adding explicit boundary conditions to the model (i.e.
considering {\sf 1-FPN(BC)}-specifications) makes the specification 
language provable
more ``powerful'' in that problems certain problems which are polynomial time
solvable for {\sf 1-FPN}-specified inputs now become {\sf PSPACE-hard}. 
In the past,
no work has been done to compare various periodic specifications in terms of
the complexity of the problems specified using such specifications. Also,
previous work has treated {\sf L}-specifications and 1-dimensional periodic 
specifications separately. But as our results show these specifications are
quite closely related and it is possible to derive hardness or easiness results
for several problems for the specifications in a unified way.

As shown in other recent paper, 
succinctly specified satisfiability problems have proven useful in 
proving additional {\sf PSPACE}-hardness results for 
succinctly specified problems much in the same way as their 
standardly specified counterparts have been used to prove the 
{\sf NP}-hardness of  standardly specified problems.




\iffalse
%%%%%%%%%%%%%%%%%%%%%%%%%%%%%%%%%%%%%%%%%%%%%%%%%%%%%%%%%%%%
%%%% The ack part appears in the main tex file.
\vspace*{.2in}

\noindent
{\bf Acknowledgement:} We thank
Professors Ken Regan (SUNY-Buffalo) and Eric Allender (Rutgers University)
for constructive suggestions on an earlier draft of this paper.
We also thank Ashish Naik (University of Chicago) and 
Sandeep Shukla (SUNY-Albany) for their help and suggestions during the course
of writing this paper.
%%%%%%%%%%%%%%%%%%%%%%%%%%%%%%%%%%%%%%%%%%%%%%%%%%%%%%%%%%%%
\fi
