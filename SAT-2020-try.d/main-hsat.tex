% Paper on evolution of similar configurations
%%%%%%%%%%%%%%%%%%%% author.tex %%%%%%%%%%%%%%%%%%%%%%%%%%%%%%%%%%%
%
% sample root file for your "contribution" to a contributed volume
%
% Use this file as a template for your own input.
%
%%%%%%%%%%%%%%%% Springer %%%%%%%%%%%%%%%%%%%%%%%%%%%%%%%%%%

\documentclass{svproc}

% RECOMMENDED %%%%%%%%%%%%%%%%%%%%%%%%%%%%%%%%%%%%%%%%%%%%%%%%%%%

% to typeset URLs, URIs, and DOIs
\usepackage{url}
\def\UrlFont{\rmfamily}

% choose options for [] as required from the list
% in the Reference Guide

\usepackage{mathptmx}       % selects Times Roman as basic font
\usepackage{helvet}         % selects Helvetica as sans-serif font
\usepackage{courier}        % selects Courier as typewriter font
\usepackage{type1cm}        % activate if the above 3 fonts are
                            % not available on your system
%
\usepackage{makeidx}         % allows index generation
\usepackage{graphicx}        % standard LaTeX graphics tool
                             % when including figure files
\usepackage{multicol}        % used for the two-column index
\usepackage[bottom]{footmisc}% places footnotes at page bottom

% see the list of further useful packages
% in the Reference Guide

%%%%%% Packages and environments that we need for the paper. %%%%%%

\usepackage[ruled,linesnumbered]{algorithm2e}
\usepackage{enumitem} %\begin{itemize}[leftmargin=*]

\usepackage{tikz}
%\usepackage{times}  
\usepackage{color}  
\usepackage{wrapfig}  
%\usepackage{subfig}

\usepackage{cite}
\usepackage{amsmath, amssymb} 
%% \usepackage{amsthm}  -- Can't use this.
\usepackage{mathtools}
\usepackage{multirow}
%\usepackage{wrapfig}

\iffalse
%%%%%%%%%%%%%%%%%%%%%%%%%%%%%%%%%%%%%%%%%%%%%%%%%%%%
\newtheorem{theorem}{Theorem}[section]
\newtheorem{lemma}[theorem]{Lemma}
\newtheorem{corollary}[theorem]{Corollary}
\newtheorem{fact}[theorem]{Fact}
\newtheorem{claim}[theorem]{Claim}
\newtheorem{observation}[theorem]{Observation}
\newtheorem{definition}[theorem]{Definition}
\newtheorem{example}[theorem]{Example}
\newtheorem{proposition}[theorem]{Proposition}
%%%%%%%%%%%%%%%%%%%%%%%%%%%%%%%%%%%%%%%%%%%%%%%%%%%%
\fi


\newcommand*\circled[1]{\tikz[baseline=(char.base)]{
                \node[shape=circle,draw,inner sep=.5pt] (char) {\small #1};}}

\DeclareMathOperator*{\errlh}{\mathrm{err}_v^{0}}
\DeclareMathOperator*{\errhl}{\mathrm{err}_v^{1}}
\DeclareMathOperator*{\expect}{\mathbb{E}}
\DeclareMathOperator*{\argmax}{\arg\max}
\DeclareMathOperator*{\err}{\mathrm{err}}
\DeclareMathOperator*{\errv}{\mathrm{err}_v}
\DeclareMathOperator*{\errk}{\mathbf{err}}

\newcommand{\comment}[1]{{\color{red}(AA: #1)}}


\iffalse
%%%%%%%%%%%%%%%%%%%%%%%%%%%%%%%%%%%%%%%%%%%%%%%%%%%%%%%%%%%%%%
%%%% For making subject index -- Needed for the Complex Networks conference.

\makeindex             % used for the subject index
                       % please use the style svind.ist with
\makeindex             % used for the subject index
                       % please use the style svind.ist with
                       % your makeindex program

%%%%%%%%%%%%%%%%%%%%%%%%%%%%%%%%%%%%%%%%%%%%%%%%%%%%%%%%%%%%%%%%%%%%%%%%%%%%%%%%%
\fi

%\title{Using~SAT~Solvers~to~Examine~the~Evolution~of
%Similar~Configurations in~Graph~Dynamical~Systems:
%Analytical~and~Experimental~Results}
\title{Dichotomy Theorems for Hierarchical and 1-Dimensional Periodically Specified Boolean Constraint Satisfaction problems with Applications}
\titlerunning{Evolution of Similar Configurations} 
% Use \titlerunning{Short Title} for an abbreviated version of
% your contribution title if the original one is too long
\authorrunning{Hunt~et~al.}

\author{ Madhav V. Marathe 
       \and{~} Venkatesh Radhakrishnan, 
       \and{~} Harry B. Hunt III
       \and  {~}S. S. Ravi
        \and {\newline} Daniel J. Rosenkrantz 
        \and {~}Richard E. Stearns %%{\vspace*{-1.7in}}
}
\institute{
           Madhav V. Marathe: University of Virginia. 
            \email{marathe@virginia.edu}.  \and
           H.B. Hunt III, S. S. Ravi, Daniel J. Rosenkrantz, Richard E. Stearns: 
           University of Virginia and University at Albany -- SUNY.~
           \email{\{hunt,ssravi0,~drosenkrantz,~thestearns2,venky\}@gmail.com}.~
}

\begin{document}
\maketitle



%%%%%%%%%%%%%%%%%%%%%%%%%%%%%%%%%%%%%%%%%%%%%%%%%%%%%%%%%%%%%%%%%%%%%%%%%

%% \graphicspath{{./figs/}}


\newcommand{\cnp}{\textbf{NP}}
\newcommand{\aacomment}[1]{{\textcolor{magenta}{(AA: #1)}}}
\newcommand{\sr}{\lambda_{\max}}
\newcommand{\QED}{\hfill\rule{2mm}{2mm}}

\newcommand{\classp}{\textbf{P}}
\newcommand{\classnp}{\textbf{NP}}

\newcommand{\npc}{\textbf{NP}-complete}
\newcommand{\nph}{\textbf{NP}-hard}
\newcommand{\peqnp}{\mbox{\textbf{P} $=$ \textbf{NP}}}
\newcommand{\pneqnp}{\mbox{\textbf{P} $\neq$ \textbf{NP}}}
\newcommand{\opti}{\mbox{\textrm{OPT}$(I)$}}

\newcommand{\algt}{\mbox{$\mathcal{A}_T$}}
\newcommand{\algb}{\mbox{$\mathcal{A}_B$}}

\newcommand{\calc}{\mbox{$\mathbb{C}$}}
\newcommand{\calcp}{\mbox{$\mathbb{C'}$}}

\newcommand{\tclass}{\mathbb{T}_G}
\newcommand{\tvect}{T}
\newcommand{\dist}{\mathcal{D}}
\newcommand{\trans}{\mathbb{F}}

\newcommand{\cals}{\mbox{$\mathbb{S}$}}
\newcommand{\phasesp}{\mbox{$\mathbb{P}_{\mathbb{S}}$}}

\newcommand{\bbb}{\mbox{$\mathbb{B}$}}


\newcommand{\calcone}{\mbox{$\mathbb{C}_{1}$}}
\newcommand{\calctwo}{\mbox{$\mathbb{C}_{2}$}}
\newcommand{\calconep}{\mbox{$\mathbb{C}_{1}^{'}$}}
\newcommand{\calctwop}{\mbox{$\mathbb{C}_{2}^{'}$}}

\newcommand{\calco}{\mbox{$\mathbb{C}_{1}$}}
\newcommand{\calcz}{\mbox{$\mathbb{C}_0$}}
\newcommand{\calci}{\mbox{$\mathbb{C}_i$}}
\newcommand{\calcipo}{\mbox{$\mathbb{C}_{i+1}$}}
\newcommand{\calct}{\mbox{$\mathbb{C}_{t}$}}
\newcommand{\calctmo}{\mbox{$\mathbb{C}_{t-1}$}}
\newcommand{\dcare}{\texttt{x}}

\newcommand{\calh}{\mbox{$\mathbb{H}$}}
\newcommand{\calf}{\mbox{$\mathbb{F}$}}

\newcommand{\genprob}{\mbox{\textsc{CSC}}}

%% Notation for predecessor set.
\newcommand{\predset}[1]{\mbox{$\Pi(#1)$}}
\newcommand{\successor}[1]{\mbox{$\sigma(#1)$}}



%%%%%%%%%%% End of packages and environments needed. %%%%%%%%%%%

%% \maketitle
% Include a short abstract here (100-300 words):
%%\abstract{

        Generalized CNF Satisfiability, Local Reductions and 
            Complexity of  Succinctly Specified Problems

M.V. Marathe, H.B. Hunt III, R.E. Stearns and V. Radhakrishnan


We study the complexity and efficient approximability of various decision,
counting and optimization problems when instances are specified using 

(1) the 1-dimensional finite periodic narrow specifications of 
Wanke \cite{Wa93}, 

(2) the 2-way infinite 1-dimensional narrow
periodic (sometimes called dynamic) specifications 
of Karp and Orlin  et al. \cite{KMW67,Or82a},  and 


(3) the hierarchical specification language of Lengauer et al. 
Le86,LW87a,Le88,Le89,LW92.



We outline how  
generalized CNF satisfiability problems 
and {\bf local} reductions
can be used to obtain {\bf both} hardness and easiness results for
a number of decision, counting, optimization and approximate optimization
problems when instances are specified as in (1), (2) or (3). 

As corollaries we obtain a number of new
PSPACE-hardness and  #PSPACE-hardness results
and  a number of new
polynomial time approximation algorithms for natural PSPACE-hard 
optimization problems. In particular assuming P <> PSPACE, 
we characterize
completely the complexities of the generalized CNF satisfiability problems
SAT(S) of Schaefer Sc78, when instances are specified
as in (1), (2) or (3).
Our results answer  open questions raised in  CF+93,La89,LW92,Or84b
and significantly extend the results in LW92,HMS94,Or82a,Zu93.



%%}

%%\vspace*{-0.25in}

\section{Introduction}
\label{sec:intro}

Large objects with a highly repetitive structure can often be represented
succinctly.
In the past, several methods have been proposed to succinctly 
represent such objects.
Here we consider two of these methods, namely
(i) hierarchical specifications  
\cite{Ga82,GW83,LW92,BOW83,RH93,LW93,MH+93a,MH+94,marathe1995complexity,Wa86},
and (ii) periodic specifications 
\cite{CM91,HW92,IS87,KO91,KS88,Or82a,Wa93}.
 


Hierarchical specifications are useful in describing large scale systems with
very regular structures. 
Using hierarchical specifications, the overall design of an object can be 
partitioned into the design of a collection of modules;
this is a 
much more manageable task compared to producing a complete design in one step.
Such a top down (or hierarchical design) approach 
also facilitates the development of computer aided design (CAD) systems, since
low-level objects can be  incorporated into libraries and can thus be made
available as submodules to designers of large scale  objects.
Other areas where hierarchical specifications have found applications are
VLSI layout \cite{Le82,Le86,Le90,HLW92,HW92,RH93},  
finite element analysis, software engineering and Datalog queries 
(see \cite{HLW92,Ma94,marathe1995complexity,bottcher2020simulation,goller2005fixpoint,brenguier2012comparison,lohrey2012model,tapken1999implementing,AC94} 
and the references therein).  Recently, there has been interest in
inferring the inherent hierarchy that graphs might exhibit from the
standpoint of graph neural networks \cite{ying2018hierarchical}.

Periodic specifications
can also be used to define large scale systems with highly regular structures.
Using  periodic specifications, 
large objects are  described as repetitive connections of a basic module.
Frequently, the modules are connected in a straight line 
but the basic modules can also  be repeated in two or higher
dimensional patterns. Periodic specifications are also used to 
model time variant  problems,  where the constraints
or demands  for any one period is the same as those for preceding or 
succeeding periods. 
Periodic specifications 
have applications in such diverse areas  as  transportation planning
\cite{Or82a,HLW92,marathe1995complexity,drucker2019cyclic,ho2015cyclic},
parallel programming \cite{HLW92,KMW67} wireless networks 
\cite{andreou2002radiocoloring,fotakis2006radiocolorings} and 
VLSI design \cite{IS87,IS88}. 
See \cite{chen2003periodic,chen2005periodic} for a good discussion and 
references on periodic constraint satisfaction problems.

Orlin has studied periodically specified problems over an infinite 
horizon; i.e the objects considered are infinite \cite{Or82a}. 
Ford and Fulkerson \cite{FF58} and Wanke \cite{Wa93} have 
studied the finite horizon
versions of periodic problems and Gale \cite{Ga59} has studied problems for
periodic specifications with fixed starting points. All the above mentioned
studies are for 1-dimensional periodic specifications.
Other 
researchers have studied 2-dimensional and more generally $d$-dimensional
periodic specifications; 
see \cite{drucker2019cyclic,ho2015cyclic,CM91,IS87,KO91,KS88,Wa93}. 

An important feature of  both hierarchical and periodic 
specifications is that they
can be much  more concise in describing  objects than standard  specifications.
In general, the size of an object can be \emph{exponential} in the
size of its periodic or hierarchical specifications.
Since, the  complexity of solving a problem 
is usually measured as a function of the size of the specification of the 
problem instance,  
the complexity of a problem when instances are specified using
standard descriptions can
be  different than the complexity of the same problem 
when instances are specified hierarchically or periodically.
For example, while the {\sf 3-COLORING} problem is 
{\sf NP-}complete when the graphs are represented by adjacency matrices
or by adjacency lists \cite{GJ79}, it is 
{\sf PSPACE}-complete when instances are specified
hierarchically or periodically \cite{LW92,Or84b}.
On the other hand, the {\sf 2-COLORING} problem is solvable in polynomial time
{\em even} when instances are specified by the hierarchical
specifications of Lengauer et al. \cite{LW92,LW87a} or by  the 
1-dimensional periodic specifications of Orlin \cite{Or82a}.




In this paper, we study the complexity of combinatorial, graph and 
generalized {\sf CNF} satisfiability problems,
when instances are specified in any of the following ways:

\begin{enumerate}
\item
the 1-dimensional finite periodic narrow specifications (denoted by
{\sf 1-FPN}-specifications) of Ford and Fulkerson \cite{FF58}
and of Wanke \cite{Wa93}; 
\item
the 1-dimensional finite periodic narrow specifications with explicit 
boundary conditions (denoted by {\sf 1-FPN(BC)}-specifications)
of Gale \cite{Ga59} and others; 
\item
the 2-way infinite 1-dimensional narrow
periodic (sometimes called dynamic) specifications  (denoted by
{\sf 1-PN}-specifications) 
of Karp et al. and Orlin \cite{KMW67,Or82a}; and 
\item
the hierarchical specifications (denoted by {\sf L}-specifications) of 
Lengauer \cite{LW87a}. 
\end{enumerate}

\noindent
Let $\Pi$ be a problem, whose instances are specified {\em non-succinctly} 
using one of the standard specifications in the literature. For example,
instances of {\sf CNF} satisfiability problems can be specified non-succinctly
by {\sf CNF} formulas and by sets of clauses, with each clause being a set of 
literals.
Here, we use {\sf 1-FPN-}, {\sf 1-FPN(BC)-}, {\sf 1-PN-} and {\sf L-}$\Pi$ to
denote the problem $\Pi$ 
when its instances are specified succinctly by {\sf 1-FPN-}, 
{\sf 1-FPN(BC)-}, {\sf 1-PN-} and {\sf L-} specifications, respectively. 
We use $\alpha$-$\Pi$ to denote the problems {\sf 1-FPN-}, {\sf 1-FPN(BC)-},
 {\sf 1-PN-} and {\sf L-}$\Pi$; and we use {\em succinct} specification 
to mean {\sf 1-FPN-}, {\sf 1-FPN(BC)-}, {\sf 1-PN-} and {\sf L-}specification. 
Thus for example, {\sf 1-FPN-3SAT} denotes the problem {\sf 3SAT}
when instances are specified  by {\sf 1-FPN-}specifications; and 
$\alpha$-{\sf 3SAT} denotes the problems {\sf 1-FPN-}, {\sf 1-FPN(BC)-}, 
{\sf 1-PN-} and {\sf L-3SAT}.

\smallskip

Our results are summarized in the next section.
                 %%% Introduction.

\input{preliminaries}         %%% Preliminaries.

\input{analysis}               %%% Analytical Results 

\input{experiments}            %%% Experimental results.

\section{Conclusions}\label{sec:conclusion}

We proved a translation theorem which allowed us to relate the
{\sf L-} and {\sf 1-FPN-}specification languages. 
The main appeal of the translation
theorem was in the corollaries; namely the various polynomial time
results and {\sf PSPACE}-hardness results for {\sf L}- and 
{\sf 1-FPN}-specified problems
We then investigated the complexity of an infinite collection of 
satisfiability problems SAT(S) when instances are specified by 
{\sf L-, 1-FPN, 1-FPN(BC)} or {\sf PN}-specification. 
We completely characterized the
complexity of the problems $\alpha$-{\sf SAT(S)}. In particular, 
we  showed that either the problems belong to
{\sf P} or they are {\sf PSPACE}-complete. 
Moreover, in conjunction with our translation theorem and known local 
reduction they provide alternative and unified {\sf PSPACE}-hardness proofs for
several {\sf L}- or 
{\sf PN}-specified problems considered in \cite{LW92,Or82a}.
We also prove the {\sf PSPACE}-hardness of several other 
{\sf L-, 1-FPN, 1-FPN(BC)} or {\sf PN}-specified problems.  
%Table 1 contains example of the results obtained in this paper.



Another feature of the results is to point out the relative 
economy of descriptions of
the variants of 1-dimensional narrow periodic specifications. For instance our
results show that adding explicit boundary conditions to the model (i.e.
considering {\sf 1-FPN(BC)}-specifications) makes the specification 
language provable
more ``powerful'' in that problems certain problems which are polynomial time
solvable for {\sf 1-FPN}-specified inputs now become {\sf PSPACE-hard}. 
In the past,
no work has been done to compare various periodic specifications in terms of
the complexity of the problems specified using such specifications. Also,
previous work has treated {\sf L}-specifications and 1-dimensional periodic 
specifications separately. But as our results show these specifications are
quite closely related and it is possible to derive hardness or easiness results
for several problems for the specifications in a unified way.

\iffalse
%%%%%%%%%%%%%%%%%%%%%%%%%%%%%%%%%%%%%%%%%%%%%%%%%%%%%%%%%%%%
%%% Decided to delete the following paragraph.

As shown in other recent paper, 
succinctly specified satisfiability problems have proven useful in 
proving additional {\sf PSPACE}-hardness results for 
succinctly specified problems much in the same way as their 
standardly specified counterparts have been used to prove the 
{\sf NP}-hardness of  standardly specified problems.


%%%% The ack part appears in the main tex file.
\vspace*{.2in}

\noindent
{\bf Acknowledgement:} We thank
Professors Ken Regan (SUNY-Buffalo) and Eric Allender (Rutgers University)
for constructive suggestions on an earlier draft of this paper.
We also thank Ashish Naik (University of Chicago) and 
Sandeep Shukla (SUNY-Albany) for their help and suggestions during the course
of writing this paper.
%%%%%%%%%%%%%%%%%%%%%%%%%%%%%%%%%%%%%%%%%%%%%%%%%%%%%%%%%%%%
\fi
            %%% Conclusions and future work.

\smallskip

\noindent
\textbf{Acknowledgments:}~
This work has been partially supported by
NSF Grants  ACI-1443054 (DIBBS), 
IIS-1633028 (BIG DATA),
CMMI-1745207 (EAGER) and
IIS-1908530.
%The U.S. Government is authorized to reproduce and
%distribute reprints for Governmental purposes notwithstanding
%any copyright annotation thereon.
%%%%%%%

%%\vspace*{-.30in}

\bibliographystyle{splncs03}
%% \bibliographystyle{plain}
\bibliography{refs}

\end{document}
